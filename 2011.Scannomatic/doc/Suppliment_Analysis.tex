\documentclass{article}
%set spelllang=en_GB.UTF-8
\title{Supplimentary Material: Analysis}
\author{Martin Zackrisson}
\date{\today}

\begin{document}
\maketitle
\begin{abstract}
In this section the analysis module of Scannomatic will be described.
The Scannomatic GUI is described elsewhere and will thus only be touched upon
when relating to the analysis procedure.
The analysis is run as a shell-script such that Scannomatic can execute it as
a separate process, allowing for multi-threading. 
The analysis algoritm first deterimines the position of the ficture in the image,
then using ficture calibration information to plate by plate excise them and 
analysing them seperately. 
Prior to plate analysis the gray-scale is analysed using fixture settings. 
The plates' pixel-values are transformed to the gray-scale, using interpolation.
Then these values are transformed to a cell count estimate-scale if such a 
calibration has been performed. 
The pinning matrix position is deterimined for each plate and each grid cell of
the matrix is analysed seperately.
The grid-cell is considered to have three features: cell, background and blob. 
Cell referres to the entire grid-cell, blob to the pixels where the colony has
been detected and background to those pixels that have been determined to be the
agar without growth.
For each of these several measures are done and the result of the analysis is
compiled into an xml-file.
Each step will be described in detail below.
\end{abstract}

\section{Running project.py}

\subsection{Pre-requisits}
For the analyis-script \texttt{project.py} requires a project run log-file, 
which is automatically created when a project is run in Scannomatic. 
If such file is lacking a new can be generated using the script 
\texttt{log\_file\_maker.py}.

\subsection{Command-line Syntax}

\section{Fixture Position}

\section{'Gray-scale Space' to 'Cell Estimate Space'}
To normalize the inter-scan noise that could arrise from lighting settings
around the scanner, variations in the scanner lamp's temperature profile, and
scanner sensor stochastics, the Kodak Q-60 Color Input Target was included
in the fixture.
The eight bit gray-scale image aquired from the scanner is represented as a 
two-dimensional datastructure with X, Y coordinates, where the pixel values 
at each position can be seens as describing the position in a third dimension,
Z. 
The Kodak Q-60 Color Input Target has a color neutral, hence called gray-scale,
section. 
The inter-scan noise is resolved by extracting the average the Z-values of each
section in the gray-scale in \emph{Gray-scale Space} (orignial image format).
Each section of the gray-scale has its denoted target value.
A third degree polynomial is fitted to the measured data-pairs (Kodak values,
original image values) using least square method.
As the color target is continiously decreasing/increasing the third degree
polynomial will not over-fit, while second degree polynomials fail to capture
the acceleration at the ends of the curve.
A further argument for fitting a polynomial rather than interpolating the 
points is that the former is less sensitive to measurement variations of 
individual points that arises from the fact that the pixel value axis is a
quantum axis of 256 possible steps.
The gray-scale of the Kodak Q-60 Color Input Target only hold 24 sections,
including the empty section, which would enhace such effects.
Thus, to not pass the information through a five bit filter, fitting a 
third degree polynomial, and using this curve as means to transform the
data from a discrete \emph{Gray-scale Space} to a continious 
\emph{Kodak Value Space}, the richness of information is conserved.

The \emph{Kodak Value Space} represents an intermediate stage where the 
inter-scan stochastics have been minimized. 
However, this is not the sought final representation of the data, as it is
reasonable to assume that this space is a poor representation of the number
of cells stacked in the Z-dimension in each pixel.
Especially at higher densities, when the light will be expected to have
passed through several cells before being measured, a non-linear relationship
between values from \emph{Kodak Value Space} and independet cell count
estimates is expected.
To this end, a dilusion series experiment was devised (described elsewhere).
Two measueres were used to estimate the cell count from each colony, FACS
measures and OD measures.
The Analyse One section of the GUI was used to extract data from the images.
The automatic blob detection was used whenever possible, but as the diluted
samples of the series at early time-points had less than one 
\emph{Kodak Value Space}-unit offset compared to the background, the algoritms
failed to properly detect the to the eye visible contour of the colony.
In these cases circle of appropriate size was drawn ontop of the colony and
was used to set the blob.
For each colony, the background mean in \emph{Kodak Value Space} was subtracted
from the blob mean. 
The independent cell count estimate was input to the GUI, and automatically 
devided by the blob area. 
These two values were considered paired measures of the same biological 
phenomenon, the former describing the average blob pixel value offset compared to
the background (agar) in \emph{Kodak Value Space} and the latter describing the
average estimated number of cells stacked in a blob pixel.
The FACS-based and OD-based measures where plotted against each other producing
figure ---INSERT FIGURE---, showing a good linear correlation with a slight
deviation from having a first derivative of one. 
It would imply that both measures are agreeing with the exception that one is
linearly and systematically over or underestimating the cell count compared to 
the other.
Thus, the OD-measures were used as basis for the \emph{Kodak Value Space} to
\emph{Cell Estimate Space} as these are easier experimentally to perform.
The plot of the OD-values versus the Kodak values shows, as discussed, a clear
non-linear correlation (see figure ---INSERT FIGURE---).
Further, the upper limit of detection seems not to be exhausted, and the range 
span several orders of magnitude.


\section{Grid Placement}

The algorithm to place the grid over the plate so that its intersections 
coinside with the pinned colonies uses knowledge about the pinning matrix
to infer the position. 
That is, the script needs to know the number of columns and rows used when
pinning.
Using the fixture position information and the information from the fixture
calibration, for each plate in the fixture, the area is sliced out of from the 
full image. 
The rows and the columns are detected separately, but using identical algoritms
therefore below will be explained using rows as the example.

\subsection{Spike-detection}

First the image is flattened to one dimensional signal, using the the mean of
each orthagonal vector to represent the signal at that position.
Then Otsu's [ref] threshold is calculated, which estimates a threshold for
separating two distributions, for the histogram of the entire plate. 
Since the majority of the one-dimensional signal describes two different types
of orthagonal vectors, those that hit a row of colonies and those that hit the
inter-space the signal has a clear oscilating tendency between the two types.
A bolean operation transverses the signal and records each time the signal
passes the Otsu threshold.
The middle position value between the threshold passage positions' values for
the sections that represents potential rows are recorded and referred to as 
spikes. However, since the edges of the plate gives similar signal to the 
true spikes, at this stage there are many false positives at the beginnig and
end of the signal.

\subsection{Spike Frequency and Offset}

The minimum number of true spikes return from spike-detection using standard
pinning procedures is 16.
This, with reasonable fixture calibration produces more than two-fold more 
true positive spike signals than false positives. 
The standard pinning procedures produces colonies at precise intervals.
The spike-detection give close to zero false positives and false negatives 
for the region of the signal that confers to where the colonies are.
This makes the median inter-spike distance a good measure of the frequency
of true spikes.

All possible offsets lesser than the frequency are evaluated. 
For each possible offset the squared distances to the most proximate 
ideal peak for all detected potential peaks are added to an array. 
The $n$ (where $n$ represents the expected number of signals e.g. numbers
of rows pinned on the plate) smallest values are summed and recorded as a 
quality-measure of that offset. 
Only keeping $n$ distances is nessesary due to unfortunate design of the
pinning head making the plate edge lie very closely at a regular interval
from the outermost rows and columns. 
The plate's edges also, in most cases, gives several peaks making these 
count too much during offset if all peaks are taken into account.
Finally the offset that had the best quality-score is reported.
Manual inspection of algorithms output shows that it typically ranks the
best offsets such that if e.g. 15 was the favoured, then it is followed
by 14 and 16 and these by 13 and 17. 
That is, it is highly robust, and is unlikely to be significantly shifted
by stocastic processes in the image capturing.

\subsection{Finding the First True Peak}

With the true peak frequency and the optimal offset decided the final
parameter is to decide at which peak of this ideal signal the first true
peak occurs.
That is where the first row is.
All potential starting peaks are evaluated, using the condition that all
$n$ peaks (where $n$ is the expected number of peaks to detect) must fit
in the sliced image section that is the plate.
For each potential starting position, the ideal peak signal is constructed.
For each peak in this ideal peak signal, the squared distance to all 
detected peaks are calculated and the closest detected peak assessed.
The closest detected peak is furthered verified so that the currently 
investigated ideal peak is its closest peak.
That is to ensure that a detected peak is only used once.
If this is true the squared distance is added to a quality score for
that starting position of the ideal signal. 
Further, the fact that a peak was detected is also counted seperately.
The best ideal signal starting position is then selected by selecting
the position for which most peaks were detected. 
If two starting-position hit equal number of peaks, then the distance
quality score is used to evaluate which to use.

\subsection{Substituting Ideal Values For Detected Values}

As a final measure the ideal peaks in the ideal peak signal are 
substituted for detected peaks if the squared distance between the ideal peak
and the measured peak is less than a threshold.
This threshold is set as the mean squared distance of all potential peaks
to the closest ideal peaks (disregarding starting position) plus three
standard deviations of that distance distribution.
Also taken into account is that a detected peak may only be used for the
ideal peak if it is the closted ideal peak.
If for an ideal peak there is no detected peak that qualifies, the ideal
peak is kept.

\subsection{The Grid-cells}

Each intersection of a row-signal and a column-signal is considered the
center of a grid-cell.
The inter-row distance is used as the height of the grid-cell and the 
inter-column distance is used as the width of the grid-cell.
Note that the size of the grid-cell is set after the first image has been
analysed (normally the last image of the time-series).
The size of the grid-cells is then kept constant throughout the whole 
analysis, while the distance between the grid-cell centers is allowed to 
vary. 
The rational for not allowing the size of the grid-cell is that, while
formulae for calculating the growth of biomass on the agar aims to 
minimize the noise from the agar, this is not perfect and changing the 
size would change the ratio of blob to background.
This would cause sufficient inter-scan noise to blur the signal in the 
beginning of the time-series.

\section{Blob, Background, Cell}

\subsection{Fallback Decteciton}

\subsection{Measures}

\section{Output}

\subsection{XML-output}

\subsection{Images and Graphs}

\end{document}
